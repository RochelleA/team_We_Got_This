\documentclass[]{article}

%opening
\title{Traffic Simulation Notes}
\author{team\_we\_got\_this}

\begin{document}

\maketitle


\section*{Intro}

In order to accurately study a real-life system, it is often necessary to first model this system by abstracting some information. Once this model has been made, one can build a simulation to make several observation of the model. Finally one analyses the information collected from the simulation in order to make inferences and suggestions \cite{sokolowski2011principles}. A traffic simulation can be classified in several ways - microscopic, mesoscopic and macroscopic simulations. Another distinction of a traffic simulator is discrete or continuous. In a discrete system the variables change at set intervals of time whereas in a continuous system the variables change continuously with time. 

Macroscopic traffic simulations are large traffic simulation capable of showing the effect of small changes on vast and complex networks. Whereas a microscopic traffic simulation is used to model individual movement of cars  in smaller sections of a network such as one specific intersection. These models will study parameters such as velocity and acceleration making them more suitable for study the effect of a change such as a new ramp. \cite{sokolowski2011principles}. Macroscopic and mesoscopic models are similar in their approaches as they capture traffic dynamics in lesser detail. This results in a faster and easier simulator which is more suitable for larger networks . 
On the other hand, microscope are applied to smaller areas as it better represents vehicle and driver-behaviour \cite{burghout2005hybrid}. 

\bibliographystyle{plain}
\bibliography{../references}


\end{document}
