\documentclass{article}
\begin{document}

	 We planned to first build the road and get the cars to properly drive on them with a basic intersection. Our first working model was hard coded to depend on the size of the map and used traffic lights to regulate the flow of traffic. We decided to move on from this to use a grid based model in which we would add in a roundabout. The first model would need to be adapted whenever the map changed. Also a roundabout would have been difficult to implement in this way since the cars have x- and y-coordinates. Though a grid is discrete, we would attempt to make it as continuous as possible by making the squares of the grid as small as possible. We would first implement the roundabout with no traffic lights, attempt to break it by creating a heavy flow from one direction which in turn would block cars from entering at the next junction. We would then fix this problem by adding in traffic lights to regulate the flow of cars into the roundabout. We also planned to have multiple lanes leading into and out of the roundabout as well as a timer to start and let it run for a set amount of time.	
		\newline\indent	If we finished with the necessary requirements of our simulation we planned to add in the requirements we designated as optional. In our research we came across a special type of roundabout called a spiral roundabout that has been implemented in northern London. In this type of roundabout a car need only to enter into the designated lane of the roundabout and it need never change lanes to exit. See figure (INSERT REF TO SP ROUNDABOUT DRAWING). One problem we foresaw with a normal roundabout was giving the intelligent to cars for when to change lanes and not cause accidents. The spiral roundabout was our solution to this problem. We also hoped to be able to implement emergency vehicles which could bypass the traffic lights and change lanes to get around cars in its way. Our last optional requirement was dynamic traffic lights which would depend on the number of cars coming from each junction into the roundabout.
		\newline\indent	If both the necessary and optional requirements were implemented and there was still time to be spared, we would try to implement any of the requirements we listed as 'extra.' This included different types of vehicles such as buses, places for cars to park, a zebra crossing where cars would have to stop to wait for a pedestrian to cross if one appeared. We would also try to give the cars behavior such as differing levels of recklessness or caution as well as the ability to honk their horn at a slow driver who would then speed up as a reaction.
		\newline\indent	We thought that creating categories like this would give us a step-ladder approach to first get the basics and then go for the slightly more advanced version before adding in extra elements.
		
		
\end{document}