\documentclass{article}
\usepackage{graphicx}
\graphicspath{ {Images/} }





\begin{document}
	\title{team\_we\_got\_this\\ Group Project \\ Log File\\Date:19th of January\\Time:10:00-12:00am}
	
	\maketitle


	{\bf\underline{Group Members Present}}\\\\
	Zaki, Rochelle, Kim, Nur\\\\

	

	{\bf\underline{Agenda}}\\
	
		\begin{itemize}
			\item{Set up Github account and group}
			\item{Decide on a team-leader for the group}
			\item{Organise parts/roles for each team}
			\item Organise weekly time slots- about 4-5 hours
		\end{itemize}
	
	
	
	\underline{\bf{Action Points}}\\
	\begin{itemize}
		\item Team name - we\_got\_this
		\item team/coordinator-Rochelle 
		\item Proposed weekly time slots
			\begin{itemize}
				\item Monday 10-12pm 
				\item Thursday 10-12pm
			\end{itemize} 
		\item Groups
			\begin{itemize}
				\item Back-end/Java - Nur \& Zaki
				\item GUI/Actionscript -Kim, Anton \& Rochelle 
			\end{itemize}
	\end{itemize}
	\newpage
	
	
	
	{\bf\underline{Miscellaneous Notes}}
	\begin{itemize}

		\item Main functions modes of transport we will try to include are cars and buses.
		 If we have time we will try to add pedestrians so we can make use of zebra crossings
		\item Typical functions we will need to build are go and stop \\
		\item We would like to represent our model on a matrix which we can then layer with our design. 

		\item Main functions modes of transport we will try to include are cars and buses. 
		If we have time we will try to add pedestrians so we can make use of zebra crossings
		\item Typical functions we will need to build are go and stop \\
		\item We would like to represent our model on a matrix which we can then layer with our design.

		This could help us to track where a vehicle is on a road and represent the space between vehicles. 
	\end{itemize}	

	\vspace{2cm}

	{\bf\underline{Design for the Simulation}}\\\\
	As mentioned in our previous meeting, it may not be possible to build the complete model we designed.

	 As a starting point we would like to focus our attention on the roundabout. 
	 In our initial design this was the most challenging aspect as any vehicle entering the roundabout will have to give way to vehicles already on the roundabout. 
	 Hence it will be most effective to start with this and introduce other junctions at a later date.\\

	As a starting point we would like to focus our attention on the roundabout. 
	In our initial design this was the most challenging aspect as any vehicle entering the roundabout will have to give way to vehicles already on the roundabout. Hence it will be most effective to start with this and introduce other junctions at a later date.\\


	\noindent The original roundabout has also been adapted to have 3 exits with two lanes of traffic, one entering and the other leaving the junction. 
	There is also a single lane entrance and a single lane exits at the roundabout.
	 It was necessary to change the design of the roundabout because we can now create one set of code for the two lane traffic and make minor alterations to represent their exact positions. 
	 We can also reuse the code for the single lane traffic and adapt it to represent the traffic leaving and exiting the roundabout through the single lanes. 

%	\begin{figure}[h]
%		\caption{Second Simpler Design for our System}
%		\includegraphics[width=1.4\textwidth]{SecondModelLayout}
%		\centering
%	\end{figure}


	
\end{document}
